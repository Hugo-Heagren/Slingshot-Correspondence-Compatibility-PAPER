The `slingshot' is a form of argument (of which Davidson's `Great Fact argument is an exmaple)which shows that all true sentences refer to the same thing.
This conclusion hass been used argue against fact-based and similar correspondence theories of truth, to motivate either non-facts based correspondence theories, or rejecting correspondence completely.

This essay argues that the conlcusion of the slingshot argument is compatible with facts-based and similar correspondecne theories of truth, and that in fact the intuittions which lead to correspondence theories accord well with the slingshot's conclusions when applied to truth.

I will use the term truthbearer for things which are true or false (perhaps sentences, propositions, beliefs etc.) and truthmaker for things which make truthbearers true or false (facts, states of affairs etc.).
This is to show that I am concerned with the correspondence theory in general, and not any particular formulation which commits to truthbearers or -makers being a certin thing (propsitions, facts etc.)

The `correspondence theory' is a group of theories, sharing the view that truth is a relation (`correspondence') with a worldly truthamker (a fact, a `state of affairs' etc.).
This is motivated by intuition: intuitively truth  (especially contigent truth) has to do with \emph{is the case} in the world, and true truthbearers are `made true' somehow (in this case, by corresponding with a fact).
The correspondence theory also agrees with the intuition that truth has a single nature (that of correspondence with a truthmaker).
This is monism about truth.
Althought this is often not explicitly mentioned, pluralistic theories of truth (that truth has multiple natures, or is multiply realisable) have been developed, but the correspondence theory commit to monism.

The slingshot (so-called because it is a small argument meant to bring down a great opponent) is a form of argument which shows that all true truthbearers have the same reference (as do all false ones).

The argument assumes:

	\begin{thesis} \label{srefer}
	Sentences refer (as names or definitie descriptions do).
	\end{thesis}


	\begin{thesis} \label{sameref}
	Logicaly equivaent singular terms (including sentences) have the same reference.
	\end{thesis}

	
	\begin{thesis} \label{constref}
	The reference of a complex singular term (e.g. 
a sentence) will be unchanged if a singular term whch is a part of it is replaced with another singular term with the same reference.
	\end{thesis}
	
$A$ and $B$ are any two sentences with the same truth value. 
The following sentence must refer to whatever $A$ refers to, and will be true if and only if $A$ is true.


	\begin{example} \label{setA}
	The set of all objects which are identical with themselves and (for which) $A$ obtains is identical with the set of all objects which are identical with themselves.
	\end{example}

All things are identical with themselves, so the `set of all objects which are identical with themselves and (for which) $A$ obtains' will just be the set of all things if $A$ is true, and the emtpy set otherwise.
Since the whole sentence sentence just asserts that this is in fact the same as the set of all objects, \ref{setA} is true just if the first set is the set of all objects.
Hence, \ref{setA} is true just if $A$ is true (the two are logically equivalent (\ref{sameref}).
Since \ref{setA} is logically equivalent with $A$, the two must have the same reference. 
(\ref{sameref})

The sentence 

	\begin{example} \label{setB}
	The set of all objects which are identical with themselves and (for which) $B$ obtains is identical with the set of all objects which are identical with themselves.
	\end{example}

will be true just if $B$ is true, for similar reasons to \ref{setA}.
The expression `the set of all objects which are identical with themselves and (for which) $B$ obtains' can be constructed from it's counterpart in \ref{setA}, only by substituting $B$ for $A$.
Since $B$ and $A$ have the same truth value, and the reference of the expression depends only the truth value of the sentence $A$ or $B$ in it, the sentencee's reference must be the same as it's counterpart in \ref{setA}.
Hence, \ref{setB} is constructed from \ref{setA} by substituting a singular term with the same reference, so \ref{setB} must have the same reference as \ref{setA} (which had the same reference as $A$). 
(\ref{constref})

By reversing the first step (which showed that \ref{setA} has the same reference as $A$), \ref{setB} must have the same reference as $B$.

So, $B$ refers to the same thing as \ref{setB}, which refers to the same thing as \ref{setA}, which refers to the same thing $A$.
Since the argument runs for any two sentences with the same truth value, all sentences with the same truth value must refer to the same thing.

Many objections can be made to this argument in itself, but this essay is not concerned with them.

This conclusion in itself is not incompatible with the correspondence theory.
The correspondence theory does not say that all true sentences do \emph{not} refer to the same thing.
A further premise is needed: that the correspondence theory is in some way incompatible with the conclusions of the slingshot argument:


	\begin{thesis} \label{incompatible}
	If the correspondence theory is true, then not all true sentences / truthbearers refer to / are made true by the same fact / truthmaker.
	\end{thesis}

\ref{incompatible} is motivated by the intuition that different true sentences refer to different facts, because different sentences mean or are `about' different things.
% someone in some paper had something about Davidson using the SA as RAA. Maybe quote them if I hae space?
The conclusion of the slingshot argument then shows that all true sentences are about the same thing, which is absurd!

However, this intuition is incorrect.
What a sentence means or is about is not \emph{just} what the sentence as a whole refers to, because meaning in general is not \emph{just} reference.

For example, the expression 'the colour of water' has no reference (because water is colourless), but it does not have no \emph{meaning}.

Sentences are a type of expression, and it seems unliekly that this would not be true of sentences too: though two sentences refer to the same thing, they do not necessarily \emph{mean} the same thing.

Given this, it is possible that two true sentences both refer to the same thing, but still \emph{mean} or \emph{are about} different things.

Accepting this, the slingshot argument's conclusion does not contradict correspondence theoryIn fact, it accords well with the intuition that all true sentences are true in the same way: a sentence is true just if it refers to the Great Fact, (but it's meaning is independent of this).
