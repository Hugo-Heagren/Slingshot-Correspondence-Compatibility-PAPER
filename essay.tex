The `slingshot' is a form of argument (of which Davidson's `Great Fact argument is an exmaple) which shows that all true sentences refer to the same thing.
This conclusion hass been used argue against fact-based and similar correspondence theories of truth, to motivate either non-facts based correspondence theories, or rejecting correspondence completely.

This essay argues that the conlcusion of the slingshot argument is compatible with facts-based and similar correspondecne theories of truth, and that in fact the intuittions which lead to correspondence theories accord well with the slingshot's conclusions when applied to truth.

% Exlain that the argument presente dher concerns factrs and reference, but it can be extended or apllied to correspeondence theory in general.
% Explain how, and cite for that.
I will use the term truthbearer for things which are true or false (perhaps sentences, propositions, beliefs etc.) and truthmaker for things which make truthbearers true or false (facts, states of affairs etc.).
This is to show that I am concerned with the correspondence theory in general, and not any particular formulation which commits to truthbearers or -makers being a certin thing (propsitions, facts etc.)

% Might want to format these intuitins as theorems
The `correspondence theory' is a group of theories, sharing the view that truth is a relation (`correspondence') with a worldly truthamker (a fact, a `state of affairs' etc.).
This is motivated by intuition: intuitively truth  (especially contigent truth) has to do with \emph{is the case} in the world, and true truthbearers are `made true' somehow (in this case, by corresponding with a fact).
The correspondence theory also agrees with the intuition that truth has a single nature---that of correspondence with a truthmaker.
This is `monism' about truth.
Althought this is often not explicitly mentioned, pluralistic theories of truth (that truth has multiple natures, or is multiply realisable) have been developed, but the correspondence theory commit to monism.

The slingshot (so-called because it is a small argument meant to bring down a great opponent) is a form of argument which shows that all true truthbearers have the same reference (as do all false ones).
% Link reference and truthmaking (like in the proviso about reference and truthmakers in the terminology paragroaph above).
% Exlain HOW siilar arguments can bemade. What would stay the same? what would change?
The version of the slingshot presented here is Davidsons (REFER HERE) in support of non-factsbased correspondence thoery, but similar arguments have been used against correspondence theory as a whole. 
% Davidosn says 'It is facts correspondence to which is said to make statements true', but of course such facts could be raplced with any other candidate truthmaker ((true) proppositinos, (obtains) states of affairs etc.) 
	% META: the above sentece might relace the below
This version treats sentences as truthbearers, facts as truthmakers and reference as the relevant correspondence relation.

The argument assumes:

	\begin{thesis} \label{srefer}
	Sentences refer (as names or definitie descriptions do).
                % Davison true to the facts 750 {{{
                % Since the variables replace sentences both as they feature
                % after words like 'Aristotle said that' and in truth-functional
                % contexts, the range of the variables must be entities that
                % sentences may be construed as naming in both such uses.
                % }}}
	\end{thesis}


	\begin{thesis} \label{sameref}
	Logicaly equivaent singular terms (including sentences) have the same reference.
	\end{thesis}

	
	\begin{thesis} \label{constref}
	The reference of a complex singular term (e.g. a sentence) will be unchanged if a singular term whch is a part of it is replaced with another singular term with the same reference.
	\end{thesis}
	
% AND.B the argument does NOT assume that the correspndence relation is reference. It just assumes that sentences refer, and then uses this later.
% Might want to put the expanations before the results, and try to shorten the whole thing.

$A$ and $B$ are any two sentences with the same truth value. 
The following sentence must refer to whatever $A$ refers to, and will be true if and only if $A$ is true.

	\begin{example} \label{setA}
	The set of all objects which are identical with themselves and (for which) $A$ obtains is identical with the set of all objects which are identical with themselves.
	\end{example}

All things are identical with themselves, so the `set of all objects which are identical with themselves and (for which) $A$ obtains' will just be the set of all things if $A$ is true, and the emtpy set otherwise.
Since the whole sentence sentence just asserts that this is in fact the same as the set of all objects, \ref{setA} is true just if the first set is the set of all objects.
Hence, \ref{setA} is true just if $A$ is true (the two are logically equivalent (\ref{sameref}).
Since \ref{setA} is logically equivalent with $A$, the two must have the same reference. 
(\ref{sameref})

The sentence 

	\begin{example} \label{setB}
	The set of all objects which are identical with themselves and (for which) $B$ obtains is identical with the set of all objects which are identical with themselves.
	\end{example}

will be true just if $B$ is true, for similar reasons to \ref{setA}.
The expression `the set of all objects which are identical with themselves and (for which) $B$ obtains' can be constructed from it's counterpart in \ref{setA}, only by substituting $B$ for $A$.
Since $B$ and $A$ have the same truth value, and the reference of the expression depends only the truth value of the sentence $A$ or $B$ in it, the sentencee's reference must be the same as it's counterpart in \ref{setA}.
Hence, \ref{setB} is constructed from \ref{setA} by substituting a singular term with the same reference, so \ref{setB} must have the same reference as \ref{setA} (which had the same reference as $A$). 
(\ref{constref})

By reversing the first step (which showed that \ref{setA} has the same reference as $A$), \ref{setB} must have the same reference as $B$.

So, $B$ refers to the same thing as \ref{setB}, which refers to the same thing as \ref{setA}, which refers to the same thing $A$.
Since the argument runs for any two sentences with the same truth value, all sentences with the same truth value must refer to the same thing.

Using \ref{constref}, we can infer from the sentence

	\begin{example} \label{stateA}
	the statement that A correspnds with the fact that A
	\end{example}

to the sentence:

	\begin{example} \label{stateA}
	the statement that A correspnds with the fact that B
	\end{example}

because the final $A$ in the first sentence must have the same reference as $B$.
Since this will run for any two senteces with the same truth value, all true sentences must be made true by the same truthmaker.

Davidson calls this referent the `Great Fact'. (REFER HERE!) Others have identified it with truth (or for Frege the object `the True'(REFER HERE)) or with a general state of affairs of the universe (further REFER HERE from the Stainton paper).

Many objections can be made to this argument in itself, but this essay is not concerned with them.

% Say something more like: this conclusion has been used against facts-based by introducing a further premise. 
% ...and this premise is false.
% Davidson Frege etc. {{{
% Davidson says 'there are very strong reasons, as Frege pointed out, for
% supposing that if sentences, when standing alone or in truth-functional con-
% texts, name anything, then all true sentences name the same thing. THIS WOULD
% FORCE US TO CONCLUDE THAT THE STATEMENT THAT P IS IDENTICAL WITH THE
% STATEMENT THAT Q WHENEVER P AND Q ~ARE~BOTH~TRUE~ [HAVE THE SAME TRUTH VALUE]
% presumably an unacceptable result'

% WOULD IT? Only if we assume that any two statements referring to the same thing are completely identical.

% But there are string reasons to doubt this:
% - Exactly what the identity conditions for a sentence might be is unclear. 
%   - Different identity conditions would have different consequene for the theory (I might not need this in here)
%   - If 'same wording' then different context or curcumstacnes could change the meaning
%   - if 'same meaning', then the identity conditions will ride on what meaning is, and it seems clear that meaning is not just reference, so even if two sentences do corefer, they might not comean.
% - It is unlikely that two sentences (or any expressions in general) which refer to the same thing MUST mean the same thing, because there appears to be more to meaning than just reference. (Frtege himself thought there was a further component of meanign called, sense, but there are other ways to account for meaning not JUST being reference)
%   - Examples: hesperus / phosphorus ; the colour of water (has no referent, water has no colour), but still has a meaning

% In fact, Frege himself did accept similar conclusions to the slingshot (REFER) (as Davidson notes REFER) but did not see them as a problem. Since meaning in Frege's theory is more than just reference, sentences referring to their truth values is in fact just part of Frege's theory of truth: what it is for a sentence to be true is to refer to the The True.

%}}}
This conclusion in itself is not incompatible with the correspondence theory.
The correspondence theory does not say that all true sentences do \emph{not} refer to the same thing.
A further premise is needed: that the correspondence theory is in some way incompatible with the conclusions of the slingshot argument:


	\begin{thesis} \label{incompatible}
	If the correspondence theory is true, then not all true sentences / truthbearers refer to / are made true by the same fact / truthmaker.
	\end{thesis}

\ref{incompatible} is motivated by the intuition that different true sentences refer to different facts, because different sentences mean or are `about' different things.
% someone in some paper had something about Davidson using the SA as RAA. Maybe quote them if I hae space?
The conclusion of the slingshot argument then shows that all true sentences are about the same thing, which is absurd!

However, this intuition is incorrect.
What a sentence means or is about is not \emph{just} what the sentence as a whole refers to, because meaning in general is not \emph{just} reference.

For example, the expression 'the colour of water' has no reference (because water is colourless), but it does not have no \emph{meaning}.

Sentences are a type of expression, and it seems unliekly that this would not be true of sentences too: though two sentences refer to the same thing, they do not necessarily \emph{mean} the same thing.

Given this, it is possible that two true sentences both refer to the same thing, but still \emph{mean} or \emph{are about} different things.

Accepting this, the slingshot argument's conclusion does not contradict correspondence theory but accords well with the intuition that all true sentences are true in the same way: a sentence is true just if it refers to the Great Fact, (but it's meaning is independent of this).
