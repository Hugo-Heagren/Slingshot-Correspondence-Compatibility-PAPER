The `slingshot' is a form of argument (of which Davidson's `Great Fact argument is an exmaple) which shows that all true sentences refer to the same thing.
This conclusion hass been used argue against fact-based and similar correspondence theories of truth, to motivate either non-facts based correspondence theories, or rejecting correspondence completely.

This essay argues that the conlcusion of the slingshot argument is compatible with facts-based and similar correspondecne theories of truth, and that in fact the intuittions which lead to correspondence theories accord well with the slingshot's conclusions when applied to truth.
I use Davidson's 'Great Fact' argument as an example throughout, but draw conclusions concerning the slingshot in general.

I will use the term truthbearer for things which are true or false (perhaps sentences, propositions, beliefs etc.) and truthmaker for things which make truthbearers true or false (facts, states of affairs etc.).
This is to show that I am concerned with the correspondence theory in general, and not any particular formulation which commits to truthbearers or -makers being a certin thing (propsitions, facts etc.)

% Might want to format these intuitins as theorems
The `correspondence theory' is a group of theories, sharing the view that truth is a relation (`correspondence') with a worldly truthamker (a fact, a `state of affairs' etc.).
This is motivated by intuition: intuitively truth  (especially contigent truth) has to do with \emph{is the case} in the world, and true truthbearers are `made true' somehow (in this case, by corresponding with a fact).
The correspondence theory also agrees with the intuition that truth has a single nature---that of correspondence with a truthmaker.
This is `monism' about truth.
Althought this is often not explicitly mentioned, pluralistic theories of truth (that truth has multiple natures, or is multiply realisable) have been developed, but the correspondence theory commit to monism.

The slingshot (so-called because it is a small argument meant to bring down the great opponent of correspondence theory) is a form of argument which shows that all true truthbearers have the same reference (as do all false ones).
The argument can then be exptended to show that if the correspondence theory is true, then all sentences with the same truth value must correspond to the same truthmaker.
The version of the slingshot presented here is Davidsons (REFER HERE).
Davidson assumes that facts are sentence's truthmakers, but the argument can be constructed  with any candidate truthmaker (e.g. (true) propositions, (actual) states of affairs etc.)
\parencite[752]{Davidson_1969} 

The argument assumes:

	\begin{thesis} \label{srefer}
	Sentences refer (as names or definitie descriptions do).
                % Davison true to the facts 750 {{{
                % Since the variables replace sentences both as they feature
                % after words like 'Aristotle said that' and in truth-functional
                % contexts, the range of the variables must be entities that
                % sentences may be construed as naming in both such uses.
                % }}}
	\end{thesis}


	\begin{thesis} \label{sameref}
	Logicaly equivaent singular terms (including sentences) have the same reference.
	\end{thesis}

	
	\begin{thesis} \label{constref}
	The reference of a complex singular term (e.g. a sentence) will be unchanged if a singular term whch is a part of it is replaced with another singular term with the same reference.
	\end{thesis}
	
% Might want to put the expanations before the results, and try to shorten the whole thing.
$A$ and $B$ are any two sentences with the same truth value. 
The following sentence must refer to whatever $A$ refers to, and will be true if and only if $A$ is true.

	\begin{example} \label{setA}
	The set of all objects which are identical with themselves and (for which) $A$ obtains is identical with the set of all objects which are identical with themselves.
	\end{example}

All things are identical with themselves, so the `set of all objects which are identical with themselves and (for which) $A$ obtains' will just be the set of all things if $A$ is true, and the emtpy set otherwise.
Since the whole sentence sentence just asserts that this is in fact identical with the set of all objects, \ref{setA} is true just if the first set is the set of all objects.
Hence, \ref{setA} is true just if $A$ is true---so the two are logically equivalent.
Since \ref{setA} is logically equivalent with $A$, the two must have the same reference. 
(\ref{sameref})

For similar reasons, the following sentence will be true just if $B$ is true:

	\begin{example} \label{setB}
	The set of all objects which are identical with themselves and (for which) $B$ obtains is identical with the set of all objects which are identical with themselves.
	\end{example}

The expression `the set of all objects which are identical with themselves and (for which) $B$ obtains' can be constructed from it's counterpart in \ref{setA}, only by substituting $B$ for $A$.
Since $B$ and $A$ have the same truth value, and the reference of the expression depends only the truth value of the sentence $A$ or $B$ in it, the sentencee's reference must be the same as it's counterpart in \ref{setA}.
Hence, \ref{setB} is constructed from \ref{setA} by substituting a singular term with the same reference, so \ref{setB} must have the same reference as \ref{setA}. 
(\ref{constref})

By reversing the first step (which showed that \ref{setA} has the same reference as $A$), \ref{setB} must have the same reference as $B$.

So, $B$ refers to the same thing as \ref{setB}, which refers to the same thing as \ref{setA}, which refers to the same thing $A$.
Since the argument runs for any two sentences with the same truth value, all sentences with the same truth value must refer to the same thing.

Given that both sentences refer to the same thing, using \ref{constref}, we can infer from the sentence

	\begin{example} \label{stateA}
	the statement that A correspnds with the fact that A
	\end{example}

to the sentence:

	\begin{example} \label{stateB}
	the statement that A correspnds with the fact that B
	\end{example}

% Use a footnote to describe that:
% - Davidson uses facts, but tis applies to all truthmakers (I only use facts for the sake of 
% I have used A and BE rather than Davidson's letters?

because the final $A$ in the first sentence must have the same reference as $B$.
Since this will work for any two senteces with the same truth value, all true sentences must be made true by the same truthmaker.

Davidson calls this truthmaker the `Great Fact'.
(REFER HERE!)
Others have identified it with truth (or for Frege the object `the True'(REFER HERE)) or with a general state of affairs of the universe (further REFER HERE from the Stainton paper).

Many objections can be made to this argument in itself, but this essay is not concerned with them.

This conclusion in itself is not incompatible with the correspondence theory.
The correspondence theory does not say that all true sentences do \emph{not} coorespond to the same truthmaker.
A further premise is needed: that the correspondence theory is in some way incompatible with the conclusions of the slingshot argument:

	\begin{thesis} \label{incompatible}
	(If the correspondence theory is true,) then not all true truthbearers are made true by the same truthmaker.
	\end{thesis}

\ref{incompatible} is motivated by the intuition that true sentences which correspond with the same truthmaker are similar is some way (other than just 'corresponding with the same truthmaker') that not all true sentences are.
Put another way: true sentences actually differ in some respect, but if all true sentences had the same truthmaker, then they would not so differ.
For example, meaning might be such a difference: one might think different true sentences mean different things, but sentences with the same truthmaker always mean the same thing.
The conclusion of the slingshot argument would then show that all true sentences mean the same thing, which is absurd!
Similar absurdities could be derived with other differences that sentences have, but supposedly could not have if they had the same truthmaker.

Davidson expresses a similar sentiment: that if two sentences correspond to the same thing then the statement of each is `identical' with that of the other.
But clearly not all true sentences state identical things.
Davidson takes this as a reduction to absurdity of (facts-based) correspndence theory. % someone in some paper had something about Davidson using the SA as RAA. Maybe quote them if I hae space?
\parencite[750]{Davidson_1969}

% \begin{quotation}
% there are very strong reasons \ldots\ for supposing that if sentences \ldots\ name anything, then all true sentences name the same thing.
% This would force us to conclude that the statement that p is identical with the statement that q whenever p and q [have the same truth value]; presumably an unacceptable result
% \end{quotation}

However, the above intuition, and Davidson's articuat of it are false.
Just because two sentences have the same truthmaker, they need not mean the same (as in the intuitive version), state identical things (as in Davidsons) or hae anything else `important' in common (more generally).
For example, if I have just eaten a apple and it has been my only one this week, then the following two sentences have the same truth maker:

	\begin{example} \label{yesterday} 
	I ate an apple yesterday. 
	\end{example}

	\begin{example} \label{week} 
	I have eaten an apple this week.
	\end{example}

If one accepts the slingshot argument, then this truthmaker will the be Great Fact, if not then it will be the fact (or proposition etc.) that I just ate a red apple.
Whatever it is, the truthmaker will be the same for the two sentences, but they mean very different things, and have very different contents.
They have different semantically significant parts ('yesterday' and 'this week') and different meanings (one could be true and the other false, if I had eaten an apple today, but not yesterday, so the two cannot mean the same).

Generalising this, even if all true sentences have the same truth maker, they may differ in other important respects: their semantically signficant parts and their meanings.
I do not think there is any respect in which sentences with the same truthmaker must be the same but in which sentences actually differ.
Thus it is not a problem for correspondence theory that all true sentences hae the same truthmaker.
In fact Frege--whom Davidson follows in his argument \parencite[750]{Davidson_1969}---accepted a similar conclusion withut seeing it as a problem.
\parencite[216]{Frege_1948} 


Of course, this means that all true sentences are true `together' in various ways: all true sentences will be true in the same way, in virtue of the same truthmaker, `at the same time' etc.
But this is entirely in accord with the monostic intuition which correspondce theory began with.
True sentences might mean different things or express different things, but all true sentences share a common nature in their truth.
