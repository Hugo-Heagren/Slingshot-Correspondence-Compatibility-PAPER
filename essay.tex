The `slingshot' is a form of argument which shows that all true sentences have the same reference.
This conclusion has been used argue against fact-based and similar correspondence theories of truth, to motivate either non-facts based correspondence theories, or rejecting correspondence entirely.

Against this I argue that the conclusion of the slingshot argument is compatible with fact-based and similar correspondence theories of truth.
First I will explain Davidson's `Great Fact' formulation of the slingshot, a typical example of the argument.
Secondly, I show that in its most basic form, the correspondence theory is compatible with the conclusions of this argument.
Any formulation of correspondence theory which \emph{is} incompatible must then be so because of some further commitment.
I argue that the intuitions which would motivate such commitments are false, so there is no incompatibility, and that the motivations for correspondence theory \emph{support} the slingshot in some respects.
Finally I consider the implications of the slingshot for a compatible correspondence theory and sketch the outline of such a theory.

I use the term \textit{truthbearer} for a thing which is true or false (often thought to be sentences, propositions, beliefs etc.)
A \textit{truthmaker} makes a truthbearer true or false (facts, states of affairs etc.).
This terminology is meant to be neutral on what truthmakers or truthbearers actually are, because I am defending correspondence theory in general from the slingshot in general, regardless any particular formulation of either.

Correspondence theory is a group of theories sharing the view that truth is a relation (`correspondence') to a truthmaker in the real world (a fact, a state of affairs etc.).
This is motivated intuitively: truth seems to be about what \emph{is the case} in the world, and true truthbearers seem to be `made true' somehow (e.g.\ by corresponding with a fact).

The slingshot is a form of argument showing that all sentences with the same truth value have the same reference.
It can be extended to show that if correspondence theory is true, then all sentences with the same truth value must correspond to the same truthmaker.
The version of the slingshot presented here is Davidson's.
\parencite[753]{Davidson_1969}
Davidson assumes that facts are sentences' truthmakers,\footnotemark\ but the argument can be constructed with any truthmaker (propositions, states of affairs etc.).
\parencite*[752]{Davidson_1969}
\footnotetext{
It might seem that if sentences have truthmakers, then they must be truthbearers.
% It would be good to include what Davidson HIMSELF thinks here. (does he think that sentences are true primarily or secondarily?)
% (This doesn't necessarily have to be a footnote)
This would be consistent with Davidson's theory, but a more common view is that sentences are only true or false (and so only have truthmakers) \emph{secondarily}---in virtue of \emph{primary} truthbearers that sentences express or otherwise correspond to.
On many theories, these primary truthbearers are propositions.
}

The argument assumes:
\parencite[753]{Davidson_1969}

	\begin{principle}{Reference}\label{srefer}
	Sentences refer (like names or definite descriptions do).
	\end{principle}

	\begin{principle}{Equivalence}\label{sameref}
	Logically equivalent singular terms (including sentences) refer to the same thing.
	\end{principle}

	\begin{principle}{Substitutivity}\label{constref}
	% this could maybe be made more clear?
	What a complex singular term refers to will not change if a part of it which is a singular term is substituted with another singular term with the same reference.
	\end{principle}


Given these assumptions and any sentences $P$ and $Q$ with the same truth value, the argument runs as follows.

The following sentence must refer to whatever $P$ refers to, and will be true iff $P$ is true:

	\begin{example}\label{setP}
	The set of all objects which are identical with themselves and (for which) $P$ obtains is identical with the set of all objects which are identical with themselves.
	\end{example}

All things are identical with themselves, so the `set of all objects which are identical with themselves and (for which) $P$ obtains' will just be the set of all things if $P$ is true, and the empty set otherwise.
The whole sentence says that this set is identical with the set of all objects, so \ref{setP} is true iff the first set is the set of all objects.
So \ref{setP} is true iff $P$ is true so they are logically equivalent.
Since \ref{setP} is logically equivalent with $P$, they must have the same reference (\ref{sameref}).
Hence, `$P$' can be substituted for \ref{setP}.
(\ref{constref})

Applying this conclusion to sentences about correspondence with truthmakers, we show that the statement that $P$ corresponds with the fact that $Q$.
First, correspondence theory entails the following sentence:

	\begin{example}\label{stateP}
	the statement that $P$ corresponds with the fact that $P$.
	\end{example}

From this, and the fact that $P$ can be substituted for \ref{setP}, \ref{factP} can be derived:

	\begin{example}\label{factP}
	the statement that $P$ corresponds with the fact that the set of all objects which are identical with themselves and (for which) $P$ obtains is identical with the set of all objects which are identical with themselves.
	\end{example}

Next, for similar reasons to above $Q$ is logically equivalent with:

	\begin{example}\label{setQ}
	The set of all objects which are identical with themselves and (for which) $Q$ obtains is identical with the set of all objects which are identical with themselves.
	\end{example}

The expression `the set of all objects which are identical with themselves and (for which) $Q$ obtains' can be constructed from it's counterpart in \ref{setP}, only by substituting $Q$ for $P$.
The reference of that expression depends only on the truth value of the sentence $P$ or $Q$ in it (it will be all objects if $P$ or $Q$ is true, and no objects / the empty set if $P$ or $Q$ is false).
Since $Q$ and $P$ have the same truth value, the expression with `$Q$' must have the same refer as it's counterpart in \ref{setP} (`the set of all objects which are identical with themselves and (for which) $P$ obtains').

This expression is the only difference between \ref{setP} and \ref{setQ}.
Hence, \ref{setQ} is constructed from \ref{setP} by substituting a singular term with the same reference, so \ref{setQ} must have the same reference as \ref{setP}.
(\ref{constref})
Since they have the same reference, \ref{setP} and \ref{setQ} can be substituted in \ref{factP} to produce \ref{factQ}:
(\ref{constref}) 

	\begin{example}\label{factQ}
	the statement that $P$ corresponds with the fact that the set of all objects which are identical with themselves and (for which) $Q$ obtains is identical with the set of all objects which are identical with themselves.
	\end{example}

Finally, by reversing the initial step from \ref{stateP} to \ref{factP}, we derive \ref{statePQ}:

	\begin{example}\label{statePQ}
	the statement that $P$ corresponds with the fact that $Q$.
	\end{example}

Since this will work for any two sentences with the same truth value, all true sentences must be made true by the same truthmaker. (which Davidson calls the `Great Fact'.)
\parencite[753]{Davidson_1969}
Others have identified it with truth \parencite[216]{Frege_1948} or the `kind of total state of affairs we call a world' \parencite[242]{Lewis1943-LEWTMO}.

This is compatible with correspondence theory.
Correspondence theory does not say that all true sentences do \emph{not} correspond to the same truthmaker---rather that for a truthbearer to be true just is for it to correspond to the relevant truthmaker (even if there is only one truthmaker).
Most reasonable formulations of correspondence theory would allow for the possibility of at least some different truthbearers having the same truthmakers.
(in fact any reasonable theory \emph{must} do so as the same fact can be presented in different ways.) % this can go if it has to
E.g., sentences which express different things about one aspect of the nature of an object (e.g. `the apple is red' and `the apple is not blue')	could reasonably be thought to be different truthbearers, but be made true by the same truthmaker.
A further premise is needed which makes the correspondence theory incompatible with the conclusion.

	\begin{principle}{Incompatibility}\label{incompatible}
	Not all true truthbearers are made true by the same truthmaker.
	\end{principle}

\ref{incompatible} is motivated by the thought that true sentences which correspond with the same truthmaker are similar in some way (other than \emph{just} corresponding with the same truthmaker) that not all true sentences are.
Put another way: we think intuitively that some (perhaps most) true sentences actually differ in some respect, which sentences with the same truthmaker cannot differ in.
For example, one might think different true sentences mean different things, but sentences with the same truthmaker always mean the same thing.
The conclusion of the slingshot argument then shows all true sentences mean the same thing, which is false.
Or if truthbearers were known to be structurally isomorphic or similar to their truthmakers and the structures of truthbearers varied, then they could not share a truthmaker.
Similar contradictions could be derived with other differences that truthbearers have, but supposedly could not have if they had the same truthmaker.

Davidson expresses a similar sentiment: if two sentences correspond to the same thing then the statement of each is `identical' with that of the other.
Since not all true sentences state identical things, Davidson takes this as a reduction to absurdity of facts-based correspondence theory.
\parencite[750]{Davidson_1969}

However, the above intuition and Davidson's articulation of it are false.
Just because two sentences have the same truthmaker, they need not mean the same (as in the intuitive version), state identical things (Davidson's version) or generally have anything else `important' in common.
For example, if I have just eaten an apple and it has been my only one this week, then the following two sentences have the same truth maker:

	\begin{example}\label{yesterday}
	I ate an apple yesterday.
	\end{example}

	\begin{example}\label{week}
	I have eaten an apple this week.
	\end{example}

On the slingshot argument this truthmaker will be the Great Fact.
Alternatively it will be the fact (or proposition etc.) that I just ate a apple.
Whatever it is, the truthmaker will be the same for the two sentences, but they mean different things and have different contents in other respects.
They have different semantically significant parts (`yesterday', `this week').
They mean different things: one could be true and the other false, so the two cannot mean the same.

Generalising this, even if all true sentences have the same truthmaker, they may differ in other important respects: their semantically significant parts, their meanings, etc.

I do not think there is any respect other than truth in which sentences with the same truthmaker must be the same but in which sentences actually differ. 
If there were then sentences like \ref{yesterday} and \ref{week} would have to be similar in this regard, but I cannot find any such regard in which they must be so.
They can mean different things, concern different subjects and objects, have sentential structure and content, and can have different pragmatics when uttered.
Thus it is not a problem for correspondence theory that all true sentences have the same truthmaker.
In fact Frege---whom Davidson follows in his argument \parencite[750]{Davidson_1969}---embraced a similar conclusion without seeing it as a problem.
He thought that all true sentences referred (as whole sentences) to The True, the analogue of Davidson's Great Fact, writing `[e]very declarative sentence \ldots\ is therefore to be regarded as a proper name, and its referent, if it exists, is either the true or the false'.
\parencite[216]{Frege_1948}

This is not surprising.
The expressed meaning of apparently identical sentences can differ significantly because of their pragmatics.
The final communication includes things determined by considerations beyond the apparent semantics. % (these are not determined pragmatically: I could talk here about topic-sensitive logic)
Why should sentences the same truthmaker not have significant differences in their meaning (or other semantically relevant qualities) determined by things beyond those truthmakers?

Of course this means that all true sentences are true `together' in various ways: all true sentences will be true in the same way, in virtue of the same truthmaker etc.
This is in accord with the monistic intuition which correspondence theory began with.
True sentences might mean or express different things, but all true sentences share a common nature in their truth.
So in this way this consequence accords with our intuitions too.

These conclusions will put bounds on the correspondence theory in general---or at least on any semantic theory which accepts it.

As above, it is natural to assume that the meanings of truthbearers are closely related to their truthmakers: true sentences are true because what they mean corresponds to what \emph{is}.
In particular it is easy to assume that what makes a truthbearer true and what determines (or is) its meaning are the same: a sentence about an apple means a proposition (also about the apple) and is made true by that proposition.

Sentences mean different things, so if this is true then they have many different truthmakers too.
So a correspondence theory which accepts the conclusions of the slingshot argument will have to reject this.
% meaning? meaning-maker? meaning-giver?
It would ridiculous to hold that all sentences mean the same thing, so such a theory must reject that a truthbearer's truthmaker and meaning (or meaning-maker) are identical.
\ref{yesterday} and \ref{week} support this.
They have the same truthmaker, but different meanings, so their truthmaker cannot have entirely determined their meanings.
of course this does not mean that truthbearers' truthmaker's have no effect on their meanings (in fact I think the intuition that they are close related is probably correct), just that a truthmaker alone does not entirely determine the truthbearer's meaning.

The slingshot argument also constrains \emph{what} the correspondence relation in the correspondence theory is.
A common candidate is isomorphism: a truthbearer corresponds to a truthmaker iff it is isomorphic with it.
If this were true, and all truthbearers had the same truthmaker, then they would all be isomorphic with it.
Isomorphism is transitive and symmetric, so all truthbearers would be isomorphic with each other.
But since all truthbearers are \emph{not} isomorphic with each other, this is unacceptable.

So isomorphism cannot be the correspondence relation on any compatible correspondence theory.

Instead, it must either be a relation which is symmetric and transitive, and which all truthbearers (and the one truthmaker) have alike, or which is not symmetric, not transitive, or neither and they do \emph{not} all have alike.

The former is very unlikely (though formally possible).
I think it is unlikely that there is a relation which all truthbearers and their common truthmaker all have, which is also a realistic possibility for truth-apt correspondence.

Whatever they are, truthbearers and truthmakers will be different in important respects---propositions are very different from states of affairs.
Just because truthbearers relate somehow to truthmakers, truthmakers need not similarly relate to truthbearers, so the correspondence relation is probably not symmetric.

The latter option is much more likely.

A reasonable possibility is representation: truthbearers are true when they represent the truthmaker.
Representation is non-symmetric and non-transitive, so this does not entail that truthbearers all represent each other.
This would be especially likely if the truthmaker were something like `the actual state of affairs'.
Accounts like this solve the issue: different truthbearers all correspond with the same truthmaker by representing it, but each could represent it in a different way.
This explains the difference between \ref{yesterday} and \ref{week} well.
Both represent the truthmaker (both represent the actual world), but in different ways (one with reference to yesterday, the other to the whole week).

This produces the outline of a slingshot-compatible correspondence theory.
All true truthbearers correspond with the same truthmaker.
Truthbearers' meaning is at least partially determined by something other than their truthmakers.
The correspondence relation between truthbearers and truthbearers is not isomorphism, and is probably non-symmetric.

Such a theory is not implausible, and should give pause to those who think the slingshot is a knockdown argument against the correspondence theory.
