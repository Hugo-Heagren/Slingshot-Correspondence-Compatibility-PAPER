The `slingshot' is a form of argument which shows that all true sentences have the same reference.
This conclusion has been used to argue against facts-based and similar correspondence theories of truth, to motivate either non-facts based correspondence theories, or rejecting correspondence entirely.

Against this I argue that the conclusion of the slingshot argument is compatible with facts-based and similar correspondence theories of truth.
I show that the intuitions leading to correspondence theories accord with the slingshot's conclusions, using Davidson's `Great Fact' argument as an example throughout but drawing conclusions concerning the slingshot in general.
I survey some further consequences of accepting these conclusions, responding to some other objections to correspondence theory in the process.

Since the slingshot is meant to challenge correspondence theory significantly, this paper is meant to defend it significantly.

I use the term \textit{truthbearer} for a thing which is true or false (often thought to be sentences, propositions, beliefs etc.)
A \textit{truthmaker} makes a truthbearer true or false (often thought facts, states of affairs etc.).
This terminology is meant to be neutral on what truthmakers or truthbearers actually are, because I am defending correspondence theory in general from the slingshot in general, regardless any particular formulation of either.
Correspondence theory is a group of theories sharing the view that truth is a relation (`correspondence') with a truthmaker in the real world (a fact, a state of affairs etc.).
This is motivated intuitively: truth seems to be about what \emph{is the case} in the world, and true truthbearers seem to be `made true' somehow (e.g. by corresponding with a fact).
This also agrees with the monistic intuition that truth has a single nature---here that of correspondence with a truthmaker.

The slingshot is a form of argument showing that all sentences with the same truth value have the same reference.
It can be extended to show that if correspondence theory is true, then all sentences with the same truth value must correspond to the same truthmaker.
The version of the slingshot presented here is Davidson's.
\parencite[753]{Davidson_1969}
Davidson assumes facts are sentences' truthmakers, but the argument can be constructed with any truthmaker (propositions, states of affairs etc.) \parencite[753]{Davidson_1969})

It might seem that if sentences have truthmakers, then they must be truthbearers---this would be a significant assumption.
Sentences are often said to be true or false, but might only be secondarily so, in virtue of some more primary truthmaker which they express.
This is a common view, where the primary truthbearers are propositions.
The argument assumes:
\parencite[753]{Davidson_1969}

	\begin{principle}{Reference}\label{srefer}
	Sentences refer (like names or definite descriptions do).
	\end{principle}

	\begin{principle}{Equivalence}\label{sameref}
	Logically equivalent singular terms (including sentences) refer to the same thing.
	\end{principle}

	\begin{principle}{Substitutivity}\label{constref}
	What a complex singular term refers to will not change if a part of it which is a singular term is substituted with another singular term with the same reference.
	\end{principle}

Given these assumptions and any sentences $P$ and $Q$ with the same truth value, the argument runs as follows.

This sentence must refer to whatever $P$ refers to, and will be true iff $P$ is true:

	\begin{example}\label{setP}
	The set of all objects which are identical with themselves and (for which) $P$ obtains is identical with the set of all objects which are identical with themselves.
	\end{example}

All things are identical with themselves, so the `set of all objects which are identical with themselves and (for which) $P$ obtains' will just be the set of all things if $P$ is true, and the empty set otherwise.
The whole sentence says that this set is identical with the set of all objects, so \ref{setP} is true iff the first set is the set of all objects.
So \ref{setP} is true iff $P$ is true so they are logically equivalent.
Since \ref{setP} is logically equivalent with $P$, they must have the same reference.
(\ref{sameref})
Hence, `$P$' can be substituted for \ref{setP}.
(\ref{constref})

Applying this conclusion to sentences about correspondence with truthmakers, we show that the statement that $P$ corresponds with the fact that $Q$.
First, correspondence theory entails the following sentence:

	\begin{example} \label{stateP}
	the statement that $P$ corresponds with the fact that $P$.
	\end{example}

From this, and the fact that $P$ can be substituted for \ref{setP}, \ref{factP} can be derived:

	\begin{example} \label{factP}
	the statement that $P$ corresponds with the fact that the set of all objects which are identical with themselves and (for which) $P$ obtains is identical with the set of all objects which are identical with themselves.
	\end{example}

Next, for similar reasons to above $Q$ is logically equivalent with:

	\begin{example} \label{setQ}
	The set of all objects which are identical with themselves and (for which) $Q$ obtains is identical with the set of all objects which are identical with themselves.
	\end{example}

The expression `the set of all objects which are identical with themselves and (for which) $Q$ obtains' can be constructed from it's counterpart in \ref{setP}, only by substituting $Q$ for $P$.
The reference of that expression depends only on the truth value of the sentence $P$ or $Q$ in it (it will be all objects if $P$ or $Q$ is true, and no objects / the empty set if $P$ or $Q$ is false).
Since $Q$ and $P$ have the same truth value, the expression with `$Q$' must have the same refer as it's counterpart in \ref{setP} (`the set of all objects which are identical with themselves and (for which) $P$ obtains').

This expression is the only difference between \ref{setP} and \ref{setQ}.
Hence, \ref{setQ} is constructed from \ref{setP/ by substituting a singular term with the same reference, so \ref{setQ} must have the same reference as \ref{setP}.
(\ref{constref})
Since they have the same reference, \ref{setP} and \ref{setQ} can be substituted in \ref{factP} to produce \ref{factQ}:

	\begin{example}\label{factQ}
	the statement that $P$ corresponds with the fact that the set of all objects which are identical with themselves and (for which) $Q$ obtains is identical with the set of all objects which are identical with themselves.
	\end{example}

Finally, by reversing the initial step from \ref{stateP} to \ref{factP}, we derive \ref{statePQ}:

	\begin{example}\label{statePQ}
	the statement that $P$ corresponds with the fact that $Q$.
	\end{example}

Since this will work for any two sentences with the same truth value, all true sentences must be made true by the same truthmaker. (which Davidson calls the `Great Fact'.)
\parencite[753]{Davidson_1969}
Others have identified it with truth \parencite[216]{Frege_1948} or with the actual state of affairs.

Many objections can be made to this argument in itself, but this paper is not concerned with them.

This conclusion is compatible with correspondence theory.
The correspondence theory does not say that all true sentences do \emph{not} correspond to the same truthmaker (though this would be in the spirit of the motivating intuitions).
A further premise is needed: that the correspondence theory is in some way incompatible with the conclusions of the slingshot argument:

	\begin{principle}{Incompatibility}\label{Incompatibility}
	Not all true truthbearers are made true by the same truthmaker.
	\end{principle}

\ref{Incompatibility} is motivated by the intuition that true sentences which correspond with the same truthmaker are similar is some way (other than just corresponding with the same truthmaker) that not all true sentences are.
Put another way: we think intuitively that some (perhaps most) true sentences actually differ in some respect, which sentences with the same truthmaker cannot differ in.
For example one might think different true sentences mean different things, but sentences with the same truthmaker always mean the same thing.
The conclusion of the slingshot argument then shows all true sentences mean the same thing, which is absurd! Or if truthbearers were known to be structurally isomorphic or similar to their truthmakers and the structures of truthbearers varied, then they could not share a truthmaker.
Similar absurdities could be derived with other differences that truthbearers have, but supposedly could not have if they had the same truthmaker.

Davidson expresses a similar sentiment: if two sentences correspond to the same thing then the statement of each is `identical' with that of the other.
But not all true sentences state identical things.
Davidson takes this as a reduction to absurdity of (facts-based) correspondence theory.
\parencite[750]{Davidson_1969}

However, the above intuition and Davidson's articulation of it are false.
Just because two sentences have the same truthmaker, they need not mean the same (as in the intuitive version), state identical things (Davidson's version) or generally have anything else `important' in common.
For example, if I have just eaten a apple and it has been my only one this week, then the following two sentences have the same truth maker:

	\begin{example}\label{yesterday}
	I ate an apple yesterday.
	\end{example}

	\begin{example}\label{week}
	I have eaten an apple this week.
	\end{example}

On the slingshot argument this truthmaker will the be Great Fact.
Alternatively it will be the fact (or proposition etc.) that I just ate a apple.
Whatever it is, the truthmaker will be the same for the two sentences, but they mean different things and have different contents in other respects.
They have different semantically significant parts (`yesterday', `this week').
They mean different things: one could be true and the other false if I had eaten an apple today but not yesterday, so the two cannot mean the same.

Generalising this, even if all true sentences have the same truthmaker, they may differ in other important respects: their semantically significant parts, their meanings and other I do not think there is any respect other than truth in which sentences with the same truthmaker must be the same but in which sentences actually differ.
If there were then sentences like \ref{yesterday} and \ref{week} would have to be similar in this regard, but I cannot find any such regard in which they must be so.
They can mean different things, concern different subjects and objects, have sentential structure and content, and can have different pragmatics when uttered.
Thus it is not a problem for correspondence theory that all true sentences have the same truthmaker.
In fact Frege---whom Davidson follows in his argument \parencite[750]{Davidson_1969}---accepted a similar conclusion without seeing it as a problem.
He thought that all true sentences referred (as whole sentences) to The True, the analogue of Davidson's Great Fact.
\parencite[216]{Frege_1948}

This should perhaps not be surprising.
The expressed meaning of apparently identical sentences (the same words in the same order) can differ significantly because of their pragmatics.
The final communication includes things determined by considerations beyond the apparent semantics.
Why should we be surprised if sentences which have identical truthmakers have significant differences in their meaning (or other semantically relevant qualities) determined by things beyond those truthmakers?

Of course this means that all true sentences are true `together' in various ways: all true sentences will be true in the same way, in virtue of the same truthmaker etc.
This is in accord with the monistic intuition which correspondence theory began with.
True sentences might mean or express different things, but all true sentences share a common nature in their truth.
So this consequence accords with our intuitions too.
