The `slingshot' is a form of argument (of which Davidson's `Great Fact argument is an exmaple)which shows that all true sentences refer to the same thing.
This conclusion hass been used argue against fact-based and similar correspondence theories of truth, to motivate either non-facts based correspondence theories, or rejecting correspondence completely.

This essay argues that the conlcusion of the slingshot argument is compatible with facts-based and similar correspondecne theories of truth, and that in fact the intuittions which lead to correspondence theories accord well with the slingshot's conclusions when applied to truth.

I will use the term truthbearer for things which are true or false (perhaps sentences, propositions, beliefs etc.) and truthmaker for things which make truthbearers true or false (facts, states of affairs etc.).
This is to show that I am concerned with the correspondence theory in general, and not any particular formulation which commits to truthbearers or -makers being a certin thing (propsitions, facts etc.)

The `correspondence theory' is a group of theories, sharing the view that truth is a relation (`correspondence') with a worldly truthamker (a fact, a `state of affairs' etc.).
This is motivated by intuition: intuitively truth  (especially contigent truth) has to do with \emph{is the case} in the world, and true truthbearers are `made true' somehow (in this case, by corresponding with a fact).
The correspondence theory also agrees with the intuition that truth has a single nature (that of correspondence with a truthmaker). 
This is monism about truth.
Althought this is often not explicitly mentioned, pluralistic theories of truth (that truth has multiple natures, or is multiply realisable) have been developed, but the correspondence theory commit to monism.

The slingshot (so-called because it is a small argument meant to bring down a great opponent) is a form of argument which shows that all true truthbearers have the same reference (as do all false ones).

The argument assumes:

	\begin{thesis} \label{srefer}
	Sentences refer (as names or definitie descriptions do).
	\end{thesis}


	\begin{thesis} \label{sameref}
	Logicaly equivaent singular terms (including sentences) have the same reference.
	\end{thesis}

	
	\begin{thesis} \label{}
	The reference of a complex singular term (e.g. a sentence) will be unchanged if a singular term whch is a part of it is replaced with another singular term with the same reference.
	\end{thesis}
	
	

