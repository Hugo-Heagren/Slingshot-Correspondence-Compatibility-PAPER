% General feedback from Suzy {{{

% SIGNPOSTING needs to improve. Maybe consider section headings? Try to introduce each paragrap with a brief phrase explaing how it relates to the previous one and / or the rest of the essay.

% }}}

The `slingshot' is a form of argument (of which Davidson's `Great Fact' argument is an example) which shows that all true sentences refer to the same thing.
This conclusion has been used argue against fact-based and similar correspondence theories of truth, to motivate either non-facts based correspondence theories, or rejecting correspondence completely.

Contrary to this, I argue that the conclusion of the slingshot argument is compatible with facts-based and similar correspondence theories of truth.
I show that in fact, the intuitions which lead to correspondence theories accord well with the slingshot's conclusions (when applied to truth).
I use Davidson's `Great Fact' argument as an example throughout, but draw conclusions concerning the slingshot in general.

Since the slingshot argument is meant to be a significant challenge to correpsondence theory, this essay is meant to be a significant defense of correspondence theory.
That is, a defence of the view that truth is a relation between what is said are thought and how the world is.

I will use the term `truthbearer' for a thing which is true or false (perhaps sentences, propositions, beliefs etc.)
A `truthmaker' makes a truthbearer true or false (truthmakers might be facts, states of affairs etc.).
This terminology is meant to be neutral on what truthmakers or truthbearers actually are.
This is because I am concerned with the correspondence thoery and slingshot argument in general, not any particular formulation of either.

The correspondence theory is a group of theories, sharing the view that truth is a relation (`correspondence') with a truthmaker in the real world (a fact, a state of affairs etc.).
This is motivated by intuition: intuitively truth  (especially contingent truth) has to do with \emph{is the case} in the world, and true truthbearers are `made true' somehow (in this case, by corresponding with a fact).
The correspondence theory also agrees with the intuition that truth has a single nature---that of correspondence with a truthmaker (this is `monism' about truth).

% I might need to cite SEP for the slingshot name thing.
The slingshot (so-called because it is a small argument meant to bring down a great opponent) is a form of argument which shows that all sentences with the same truth value have the same reference.
It can then be extended to show that if the correspondence theory is true, then all sentences with the same truth value must correspond to the same truthmaker.
The version of the slingshot presented here is Davidson's.
\parencite[753]{Davidson_1969}
Davidson assumes that facts are sentences' truthmakers, but the argument can be constructed with any candidate truthmaker (e.g. (true) propositions, (actual) states of affairs etc.)
% \footnote{  uncomment me when you hand in! {{{
% It might seem that if sentences have truthmakers, then they must be truthbearers---this would be a significant assumption.
% Although sentences can be true or false, they might not be `primary' or basic truthbearers.
% Instead sentences are often thought to be `secondary' truthbearers, true or false in virtue of some `primary' truthearer they are related to (e.g. a proposition or thought they express).
% } %}}}
\parencite[752]{Davidson_1969}

% make this simpler to follow by:
% 	give more examples (like in the article) for the theses.
% 		Examples of sentences referring
% 	make it more clear what P and Q are (check out the original paper and see what their examples are)
% 	bring in P and Q earlier
The argument assumes:
\parencite[753]{Davidson_1969}

	\begin{thesis} \label{srefer}
	Sentences refer (as names or definite descriptions do).
	\end{thesis}


	\begin{thesis} \label{sameref}
	Logically equivalent singular terms (including sentences) have the same reference.
	\end{thesis}


	\begin{thesis} \label{constref}
	The reference of a complex singular term (e.g. a sentence) will be unchanged if a singular term which is a part of it is replaced with another singular term with the same reference.
	\end{thesis}

$P$ and $Q$ are any two sentences with the same truth value.
The following sentence must refer to whatever $P$ refers to, and will be true if and only if $P$ is true.

	\begin{example} \label{setP}
	The set of all objects which are identical with themselves and (for which) $P$ obtains is identical with the set of all objects which are identical with themselves.
	\end{example}

All things are identical with themselves, so the `set of all objects which are identical with themselves and (for which) $P$ obtains' will just be the set of all things if $P$ is true, and the empty set otherwise.
The whole sentence just asserts that this set is identical with the set of all objects, so \ref{setP} is true just if the first set is the set of all objects.
So \ref{setP} is true just if $P$ is true---so the two are logically equivalent.
Since \ref{setP} is logically equivalent with $P$, the two must have the same reference.
(\ref{sameref})
Hence, `$P$' can be substituted for \ref{setP}. (\ref{constref})

By applying this conclusion to sentences about correspondence with a truthmaker, we can show that the statement that $P$ corresponds with the fact that $Q$.
First, the correspondence theory entails the following sentence:

	\begin{example} \label{stateP}
	the statement that $P$ corresponds with the fact that $P$.
	\end{example}

From this, and the fact that $P$ can be substituted for \ref{constref}, \ref{factP} can be derived:

	\begin{example} \label{factP}
	the statement that $P$ corresponds with the fact that the set of all objects which are identical with themselves and (for which) $P$ obtains is identical with the set of all objects which are identical with themselves.
	\end{example}

Next, for similar reasons to above the following sentence will be logically equivalent with $Q$:

	\begin{example} \label{setQ}
	The set of all objects which are identical with themselves and (for which) $Q$ obtains is identical with the set of all objects which are identical with themselves.
	\end{example}

The expression `the set of all objects which are identical with themselves and (for which) $Q$ obtains' can be constructed from it's counterpart in \ref{setP}, only by substituting $Q$ for $P$.
The reference of the expression depends only on the truth value of the sentence $P$ or $Q$ in it. (\ref{constref} )
Since $Q$ and $P$ have the same truth value then, the sentence's reference must be the same as it's counterpart in \ref{setP}.
% this v needs a bit of LOGIC and FACTS
Hence, \ref{setQ} is constructed from \ref{setP} by substituting a singular term (a sentence) with the same reference, so \ref{setQ} must have the same reference as \ref{setP}.
(\ref{constref})
Hence, \ref{setQ} is logically equivalent with \ref{setP}, so the two can substituted in \ref{factP} to produce \ref{factQ}:
% this ^ needs a bit of LOGIC and FACTS


	\begin{example} \label{factQ}
	the statement that $P$ corresponds with the fact that the set of all objects which are identical with themselves and (for which) $Q$ obtains is identical with the set of all objects which are identical with themselves.
	\end{example}

Finally, by reversing the initial step from \ref{stateP} to \ref{factP}, we can derive \ref{statePQ}:

	\begin{example} \label{statePQ}
	the statement that $P$ corresponds with the fact that $Q$.
	\end{example}

Since this will work for any two sentences with the same truth value, all true sentences must be made true by the same truthmaker.
Davidson calls this truthmaker the `Great Fact'.
\parencite[753]{Davidson_1969}
Others have identified it with truth (or for Frege the object `the True'\parencite[216]{Frege_1948}) or with a general state of affairs of the universe % reference here to Neale paper (see SEP entry)

Many objections can be made to this argument in itself, but this essay is not concerned with them.

Althought it might seem so, this conclusion in itself is not incompatible with the correspondence theory.
The correspondence theory does not say that all true sentences do \emph{not} correspond to the same truthmaker (though to say so would be in the spirit of the intuitions which motivate it).
A further premise is needed: that the correspondence theory is in some way incompatible with the conclusions of the slingshot argument:

	\begin{thesis} \label{incompatible}
	(If the correspondence theory is true) then not all true truthbearers are made true by the same truthmaker.
	\end{thesis}

\ref{incompatible} is motivated by the intuition that true sentences which correspond with the same truthmaker are similar is some way (other than just `corresponding with the same truthmaker') that not all true sentences are.
Put another way: intuitively, true sentences actually differ in some respect, but if all true sentences had the same truthmaker, then they would not so differ.
For example, meaning might be such a difference: one might think different true sentences mean different things, but sentences with the same truthmaker always mean the same thing.
The conclusion of the slingshot argument would then show that all true sentences mean the same thing, which is absurd!
Similar absurdities could be derived with other differences that sentences have, but supposedly could not have if they had the same truthmaker.

Davidson expresses a similar sentiment: that if two sentences correspond to the same thing then the statement of each is `identical' with that of the other.
But clearly not all true sentences state identical things.
Davidson takes this as a reduction to absurdity of (facts-based) correspondence theory. % someone in some paper had something about Davidson using the SA as RAA. Maybe quote them if I have space?
\parencite[750]{Davidson_1969}

However, the above intuition, and Davidson's articulation of it are false.
Just because two sentences have the same truthmaker, they need not mean the same (as in the intuitive version), state identical things (Davidson's version) or generally have anything else `important' in common.
For example, if I have just eaten a apple and it has been my only one this week, then the following two sentences have the same truth maker:

	\begin{example} \label{yesterday}
	I ate an apple yesterday.
	\end{example}

	\begin{example} \label{week}
	I have eaten an apple this week.
	\end{example}

If one accepts the slingshot argument, then this truthmaker will the be Great Fact, if not then it will be the fact (or proposition etc.) that I just ate a apple.
Whatever it is, the truthmaker will be the same for the two sentences, but they mean different things, and have different contents in other respects.
They have different semantically significant parts (`yesterday' and `this week') and different meanings (one could be true and the other false, if I had eaten an apple today, but not yesterday, so the two cannot mean the same).

Generalising this, even if all true sentences have the same truthmaker, they may differ in other important respects: their semantically significant parts and their meanings.
I do not think there is any respect in which sentences with the same truthmaker must be the same but in which sentences actually differ.
Thus it is not a problem for correspondence theory that all true sentences have the same truthmaker.
In fact Frege---whom Davidson follows in his argument \parencite[750]{Davidson_1969}---accepted a similar conclusion without seeing it as a problem.
\parencite[216]{Frege_1948}

Of course, this means that all true sentences are true `together' in various ways: all true sentences will be true in the same way, in virtue of the same truthmaker, `at the same time' etc.
But this is entirely in accord with the monistic intuition which correspondence theory began with.
True sentences might mean different things or express different things, but all true sentences share a common nature in their truth.
So this consequecne accord with our intuitions too.
